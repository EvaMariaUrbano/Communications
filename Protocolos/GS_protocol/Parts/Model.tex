\paragraph{}
In order to find the protocols that will rule the ground communications of the network, it has been studied 4 protocol models and suits. It has to be found the advantages and disadvantages of each one and then, assess which suits better to the porpouse. The models and suits are:
\begin{itemize}
\item OSI
\item TCP/IP
\item NetBEUI
\item IPX/SPX
\end{itemize}
\paragraph{} 
OSI and TCP/IP have a complex structure which it is ideal for large networks. Althoght, this complexity makes the protocols inefficient in small networks. On the other hand NetBEUI and IPX/SPX protocols are simpler and optimum in simple comunications. They cannot work in a large network unless adding extensions. 
\paragraph{}
In the begining of the network workint, the ground segment would not be so large as could be the space segment. Since there wont be lots of Ground Stations (3 at the begining, but it cuould increase) the big part of nodes will be the clients. If it is want to be versatile and adapt to the demand it has to be implemented protocols of the TCP/IP suit or based in the OSI model. Although these 2 will be more dificult to implement and configuate, it will be the better option to ensure a good coverege for the demand and a friendly use to the costumers.
\paragraph{}
The main diference between OSI and TCP/IP is that the first is a model and the second is a suit of potocols. OSI is structured in 7 layers, and it descives how these layers should work. It is a theorical guide for build a protocol but nobody had never implemented a complet OSI protocol. TCP/IP is structured in 4 layers and it is formed by a family of protocols which can be used in this layers. In the practice almost every network is ruled by TCP/IP protocols. 
\paragraph{}
The decicison between OSI model and TCP/IP suit could be resumed as making a protocol or use existing ones. The work of the ground segment is not really diferent of many existing systems, so the better option is to use the existing TCP/IP protocols. It has to be found which fits better to the system in every layer and adapt it if it is need.

