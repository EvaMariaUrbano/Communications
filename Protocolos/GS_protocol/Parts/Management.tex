\paragraph{}
For a adecuate work of the system there would be a real-time status control of the nodes. This system has to be implemented over a UDP. In order to have information of the status of the nodes, the SNMP (application layer) gives a solution.

\paragraph{} \textbf{SNMP}
\paragraph{} 
Simple Network Management Protocol (SNMP) is an Internet-standard protocol for collecting and organizing information about managed devices on IP networks and for modifying that information to change device behavior. Devices that typically support SNMP include routers, switches, servers, workstations, printers, modem racks and more.
\paragraph{}
In typical uses of SNMP one or more administrative computers, called managers, have the task of monitoring or managing a group of hosts or devices on a computer network. Each managed system executes, at all times, a software component called an agent which reports information via SNMP to the manager.
\paragraph{}
An SNMP-managed network consists of three key components:
\begin{itemize}
\item Managed device
\item Agent — software which runs on managed devices
\item Network management station (NMS) — software which runs on the manager
\end{itemize}

\paragraph{}
It will be implemented a management system based on SNMP. That means that it has to be hided a provider which can offer a UDP network for communicating the ground stations in real-time. 
\paragraph{}
The management system will consist in connecting the ground stations (GS) to this network and a management station (NMS). This will be placed in one of the GS, which will be the operations center. The GS will send to the NMS the status of the network in real-time. If some node fails (a GS, a satellite or a server) the NMS will reconfigurate the network to avoid the failed node. 