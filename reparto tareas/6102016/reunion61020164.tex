\documentclass[12pt,a4paper]{report}
\usepackage[utf8]{inputenc}
\usepackage[english]{babel}
\usepackage{amsmath}
\usepackage{amsfonts}
\usepackage{amssymb}
\usepackage{graphicx}
\usepackage[left=2cm,right=2cm,top=2cm,bottom=2cm]{geometry}
\author{Eva María Urbano González}
\begin{document}
\section{Tasks for the first week}
\begin{itemize}
\item Búsqueda de protocolos de comunicación tanto para satélites como para internet. Asignada a Boyan.
\item Potencia de comunicación necesaria y cálculo del tiempo de transmisión. Asignado a Josep.
\item Frecuencias que se pueden contratar y como contratarlas. Estudio de las opciones legales y frecuencia óptima. Asignado a Josep Maria. 
\item Encriptamiento de la señal. Protocolo de encriptación. Gasto y tiempo que conlleva encriptar la señal. Asignado a Eva María.
\item Estudio de antenas(número, tipo, etc.), en constelaciones de satélites. Tomar como ejemplos constelaciones como Iridium, GPS, etc. Asignado a Sergi.
\end{itemize}
\section{Ideas propuestas durante la sesión}
\begin{itemize}
\item ¿Transmisión de información por el aire hasta tener contacto con la Ground Station de destino o transmitir la información lo antes  posible al suelo y transmitir por tierra hasta el destino?
\item Órbitas que pasen todas por un punto para poder tener una sola Ground Station.
\item Viabilidad de una Ground Station en un punto cercano al polo. 
\end{itemize}
\section{Límite realización del trabajo}
Este trabajo tiene que estar colgado en github \textbf{ANTES} del Miércoles. 
\end{document}