\documentclass[12pt,a4paper]{report}
\usepackage[utf8]{inputenc}
\usepackage[english]{babel}
\usepackage{amsmath}
\usepackage{amsfonts}
\usepackage{amssymb}
\usepackage{graphicx}
\usepackage{cite}
\usepackage[left=2cm,right=2cm,top=2cm,bottom=2cm]{geometry}
\author{Eva María Urbano González\\
Josep Maria Serra Moncunill}
\title{Consultative Committee for Space Data Systems (CCSDS)}
\begin{document}
\maketitle
\section{CCSDS}
\subsection{Mission and objectives}
The Consultative Committee for Space Data Systems was founded in 1982 to develop communications and data systems standards for spaceflight. The goal of CCSDS is to enhance governamental and commercial interoperability and cross-suport, while reducing risks during the communication. It also focuses its attention on reducing the time and the project costs.  For these reasons, a research through the CCSDS standards should be done for our project. Moreover, CCSDS recommendations are routinely submitted to the International Organization for Standardization (ISO) for adoption as international standards. In 1990 the CCSDS entered into a cooperative arrangement with ISO under which the CCSDS-developet recommendations are advanced to ISO TC20/SC13 where they are progressed into ISO/CCSDS Standards. 
\subsection{Space Link Services Area}
The SLS (Space Link Services) works developing efficient space link communication systems. As we already know, a space link interconnects a spacecraft with its ground station and with other satellites. In this section is possible to find protocols like the AOS Space Data Link Protocol, which is a protocol to be used over space-to-ground, ground-to-space, or space-to-space communication links.\cite{CC2006}  
\bibliographystyle{plain}
\bibliography{CC2006}
\end{document}