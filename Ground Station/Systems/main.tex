\documentclass[12pt,a4paper]{article}

\usepackage[utf8]{inputenc}
\usepackage[english]{babel}
\usepackage{amsmath}
\usepackage{amsfonts}
\usepackage{amssymb}
\usepackage{graphicx}
\usepackage{mathtools}
\usepackage{amsfonts}
\usepackage{pdfpages}
\usepackage{hyperref}
\usepackage{cite}
\usepackage{subfig}
\usepackage[left=2cm,right=2cm,top=2cm,bottom=2cm]{geometry}
\everymath{\displaystyle}
\author{Boyan Naydenov}
\title{Boundary Value Problem \textit{BVP}. Assignment 1a}


\begin{document}
\begin{titlepage}
	\centering
	\vspace{4.5cm}
	{\scshape ESEIAAT \par}
	{\scshape\Large \par}
	\vspace{1.5cm}
	{\huge\bfseries ASTREA. Systems of the ground Segment\par}
	\vspace{10cm}
	\vspace{3cm}
	{\Large\itshape Boyan Naydenov\par}
	\vfill
	\vspace{1cm}
	\today
\end{titlepage}
\tableofcontents
\pagebreak
\section{Global Description of the system}

The Groud Segment is composed of three Ground Stations, togeter with a mission control center (MCC) which will be used for the headquarters of the company and for managing all the ground stations.

Each ground station is composed of two clearly differentiated parts:
\begin{itemize}
\item Spacecraft (S/C) control
\item Payload (P/L) control
\end{itemize}

Since the \textbf{S/C control} part is crucial, it is going to be treated independently. Hence, two systems are to be designed.

\subsection{Mission Control Center}
\subsection{Ground Station}
\subsubsection{SpaceCraft Control}
The S/C control system will be in charge of receiving and transmitting the TT\&C (telemetry and telecommand) data.

Since this part does not require a big amount of data to be handled, an UHF Antenna and treansciever are chosen the task. A diagram of the system is shown below:



This is a tested and well knonw product used by many radioamateurs around the world. In case a technical issue arises, some support from the community can be exepcted.

https://www.isispace.nl/product/full-ground-station-kit-for-vhfuhf/
The price is 41.500€ and is sold by \textbf{\textit{ISIS}}.
\subsubsection{Payload Control}
\subsubsection{Global Diagram of the system}
\end{document}
