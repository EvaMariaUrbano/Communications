\section{Propuesta Josep}
\subsection{Coverage}
I hold the view that we should offer as much coverage as possible, ideally full coverage, so that the range of clients is as broad as possible. So if it is technically feasible, I think we should intend to create a full coverage network. I agree with Eva, we do not need to create Ground Stations all around the world, it is responsability of the client, which is supposed to be a professional of the field, and therefore would have a proper Ground Station, hardware and knowledge to use and dominate it.
\subsection{Physical topology}
As it has been already said before, both mesh and star topology do not match our requirements. The option that Boyan has suggested (The Manhattan Topology) seems pretty nice, but Eva's suggestion too. In fact, they seem to be a bit similar. This seems a good direction to take. We should decide further details and maybe solve some little problems, but this is the kind of topology that I would take too. We should take into account the fact that distance between satellites might change, and be sure that in the furthest point, the communication is acceptable. We should also consider effects such as Doppler Effect (both the transmitter and the receiver are moving). This requires a further study, but this is the direction that I would definitely take. It would be nice if we could discuss it via Skype or any other communication programme, and decide exactly if Boyan's or Eva's option (though quite similar, a few differences exist).

