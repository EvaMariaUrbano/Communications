\section{Propuesta Eva}
\subsection{Coverage}
To take a decision about wheter our constellation should have global coverage or not, we have to take into account our objectives. The aim of the constellation is to provide low lattency communications between our client's satellite and the client. We can create Ground Stations and put them in adecuated places where the signal could be received. But the client, who has already a satellite, is surely to have a ground station, so the most proper thing to do will be assure him that the data could be received to they ground station with low lattency. So, in my opinion, we should have global coverage in order to satisfy the client's needs. 
\subsection{Physical topology}
Regarding the physical topology, I think there are few things that needs to be said before taking a decision. First of all, we need to be aware of the physical topologies possible for our constellation. For example, we can't use a total \textit{mesh} topology because not all the satellites will be seen by one of them. Secondly, although the \textit{star} topology seems a good option (cheap, robust, secure), we have a problem. The hub will probably need to stay in a higher orbit and will require higher power. For this reason, it will probably be bigger than a 3U CubeSat, that is the maximum we fixed in the Project Charter. Finally, in order to design our constellation and its links, it's good to know that there are problems connecting two satellits moving in opposite directions which is too expensive or even infeasible with the existing technology.\cite{Ferreira2002}. We still don't know if all our satellites will be moving in the same direction or not, but it is an important fact that if they do, we won't be able to connect them even if they are close to each other. Now, here comes my proposal: I think the best thing to do is to connect the satellites ubicated in the same plane in \textit{ring}, and then create other \textit{rings} connecting one satellite of each plane to a satellite of the other. Each satellite will be connected to 4 satellites, two of their same plane and two of different planes. To get a clear image of what I am explained, is useful to check the Teledesic physical topology.\cite{Wood2003}\\

